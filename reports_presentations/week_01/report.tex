\documentclass[12pt]{article}
\usepackage[utf8]{inputenc}
\usepackage{graphicx}
\usepackage{amsmath}
\usepackage{siunitx}
\usepackage{booktabs}
\usepackage[margin=1in]{geometry}

\title{Aerospace Senior Project Weekly Report}
\author{Your Name}
\date{Week 1 - January 17, 2025}

\begin{document}
\maketitle

\section{Week 1 Progress Report}\label{week-1-progress-report}

\subsection{Aerodynamic Analysis
Summary}\label{aerodynamic-analysis-summary}

Our latest wind tunnel tests have shown promising results. The maximum
lift coefficient achieved was 1.5, which exceeds our initial target by
15\%. The drag coefficient at cruise conditions was measured at 0.028,
indicating good aerodynamic efficiency.

\subsection{Performance Metrics}\label{performance-metrics}

Current analysis shows: - Maximum range: 2500 km - Service ceiling:
12000 m - Maximum thrust: 15000 N - Specific fuel consumption: 16.2
g/kN-s

\subsection{CFD Analysis Results}\label{cfd-analysis-results}

\begin{figure}
\centering
\includegraphics{../assets/week_01/pressure_distribution.png}
\caption{CFD Pressure Distribution}
\end{figure}

The CFD analysis reveals stable flow patterns around the wing sections.
Key observations:

\begin{enumerate}
\def\labelenumi{\arabic{enumi}.}
\tightlist
\item
  No significant flow separation at cruise angles of attack
\item
  Pressure distribution matches theoretical predictions
\item
  Wingtip vortices are well-contained by our winglet design
\end{enumerate}

\subsection{Next Week's Objectives}\label{next-weeks-objectives}

\begin{enumerate}
\def\labelenumi{\arabic{enumi}.}
\tightlist
\item
  Complete structural analysis of the wing box
\item
  Begin integration of propulsion system model
\item
  Validate aerodynamic coefficients with additional wind tunnel tests
\end{enumerate}

\subsection{Technical Challenges}\label{technical-challenges}

The main challenge we're facing is the trade-off between structural
weight and aerodynamic performance. Our current design shows:

\[ L/D = \frac{C_L}{C_D} = \frac{1.5}{0.028} = {{ aerodynamics.lift_coefficient / aerodynamics.drag_coefficient }} \]

This L/D ratio suggests we're on track to meet our efficiency targets,
but further optimization may be needed.

\end{document}

% presentation_template.tex
\documentclass{beamer}
\usetheme{Madrid}
\usecolortheme{whale}
\usepackage{graphicx}
\usepackage{amsmath}
\usepackage{siunitx}

\title{Aerospace Senior Project Progress}
\author{Your Name}
\date{Week 1 - January 17, 2025}

\begin{document}

\frame{\titlepage}

\frame{
\frametitle{Contents}
\tableofcontents
}

\section{Week 1 Progress Report}\label{week-1-progress-report}

\subsection{Aerodynamic Analysis
Summary}\label{aerodynamic-analysis-summary}

Our latest wind tunnel tests have shown promising results. The maximum
lift coefficient achieved was 1.5, which exceeds our initial target by
15\%. The drag coefficient at cruise conditions was measured at 0.028,
indicating good aerodynamic efficiency.

\subsection{Performance Metrics}\label{performance-metrics}

Current analysis shows: - Maximum range: 2500 km - Service ceiling:
12000 m - Maximum thrust: 15000 N - Specific fuel consumption: 16.2
g/kN-s

\subsection{CFD Analysis Results}\label{cfd-analysis-results}

\begin{figure}
\centering
\includegraphics{../assets/week_01/pressure_distribution.png}
\caption{CFD Pressure Distribution}
\end{figure}

The CFD analysis reveals stable flow patterns around the wing sections.
Key observations:

\begin{enumerate}
\def\labelenumi{\arabic{enumi}.}
\tightlist
\item
  No significant flow separation at cruise angles of attack
\item
  Pressure distribution matches theoretical predictions
\item
  Wingtip vortices are well-contained by our winglet design
\end{enumerate}

\subsection{Next Week's Objectives}\label{next-weeks-objectives}

\begin{enumerate}
\def\labelenumi{\arabic{enumi}.}
\tightlist
\item
  Complete structural analysis of the wing box
\item
  Begin integration of propulsion system model
\item
  Validate aerodynamic coefficients with additional wind tunnel tests
\end{enumerate}

\subsection{Technical Challenges}\label{technical-challenges}

The main challenge we're facing is the trade-off between structural
weight and aerodynamic performance. Our current design shows:

\[ L/D = \frac{C_L}{C_D} = \frac{1.5}{0.028} = {{ aerodynamics.lift_coefficient / aerodynamics.drag_coefficient }} \]

This L/D ratio suggests we're on track to meet our efficiency targets,
but further optimization may be needed.

\end{document}
