\documentclass[10pt,twocolumn]{article}  % Added twocolumn option
\usepackage[utf8]{inputenc}
\usepackage{graphicx}
\usepackage{amsmath}
\usepackage{siunitx}
\usepackage{booktabs}
\usepackage[margin=0.3in,columnsep=0.2in]{geometry}  % Added columnsep

% Make sections more compact
\usepackage{titlesec}
\titlespacing*{\section}{0pt}{1.2ex plus 1ex minus .2ex}{0.5ex plus .2ex}
\titlespacing*{\subsection}{0pt}{1ex plus 1ex minus .2ex}{0.3ex plus .2ex}

% Adjust list spacing
\usepackage{enumitem}
\setlist{nosep}  % Remove vertical spacing between items
\setlist{leftmargin=*}  % Align with left margin

% Reduce paragraph spacing
\setlength{\parskip}{0.2ex plus 0.1ex minus 0.1ex}
\setlength{\parindent}{0pt}

% Make title more compact
\usepackage{titling}
\setlength{\droptitle}{-1em}  % Move title up
\pretitle{\begin{center}\Large}  % Smaller title
\posttitle{\par\end{center}}
\preauthor{\begin{center}\normalsize}  % Smaller author
\postauthor{\end{center}}
\predate{\begin{center}\normalsize}  % Smaller date
\postdate{\end{center}}


\title{Week 1 Progress Report}
\author{Team Gamma}
\date{January 21, 2025}

\begin{document}
\maketitle
\section{Progress Made}\label{progress-made}

\subsection{Product Breakdown
Structure}\label{product-breakdown-structure}

\begin{itemize}
\tightlist
\item
  We began by forming an initial product Breakdown
\item
  We then used this breakdown to begin researching requirements
\end{itemize}

\subsection{Research Topics Summary}\label{research-topics-summary}

We began researching:

\begin{itemize}
\tightlist
\item
  FAA Requlations on:

  \begin{itemize}
  \tightlist
  \item
    Runways
  \item
    Hangers
  \item
    Weather conditions
  \item
    Stability
  \item
    Safety (fire suppresion and cabin dimensions)
  \end{itemize}
\item
  Comparable Jets

  \begin{itemize}
  \tightlist
  \item
    The A380 was used for many preliminary calculations, as the largest
    currently operated passenger jet.
  \end{itemize}
\item
  Geometry using ch.~5\&6 of Raymer \#\# Requirements
\item
  Since we are designing a large transport vehicle, the 14 CFR 25
  (airworthiness standards for transport category airplanes) was used to
  determine many of the FAA requirements for the plane.
\item
  Mission requirements were also categorized.
\item
  Other ``customer requirements'' were drafted from mission requirements
  and regulations. These will be used to guide trade studies for the
  determination of design requirements.
\item
  Requirements were drafted in the following categories:

  \begin{itemize}
  \tightlist
  \item
    Stability
  \item
    Structure
  \item
    Passenger loading
  \item
    Weight
  \item
    Performance
  \item
    Operational Internal Components
  \item
    Passenger components: Windows, seats, bathrooms, food preparation
  \end{itemize}
\end{itemize}

\subsection{Weight Calculations}\label{weight-calculations}

\begin{itemize}
\tightlist
\item
  Crew weight was found to be \textasciitilde4,818 lbs
\item
  Payload weight was found to be \textasciitilde331,660 lbs
\item
  Total weight was found to be \textasciitilde2,233,541 lbs
\end{itemize}

\subsection{Additional Work}\label{additional-work}

Mission diagram, project breakdown structure diagram and preliminary
function block diagrams were created.

\section{Next Week's Objectives}\label{next-weeks-objectives}

\subsection{Continued Research}\label{continued-research}

Next week, we aim to continue researching the above topics. Specific
additional research includes:

\begin{itemize}
\tightlist
\item
  Water Storage
\item
  Takeoff thrust determination (this will aid in specifying many engine
  requirements)
\item
  Taining requirements
\end{itemize}

\subsection{Trade Studies}\label{trade-studies}

We aim to begin conducting the following trade studies:

\begin{itemize}
\tightlist
\item
  Cabin and Seat configuration
\item
  Cruise thrust requirements
\item
  Takeoff/Landing thrust requirements
\item
  Alternative fuselage designs (double fuselage, flying wing, double
  decker, etc.)
\item
  Wing geometry
\item
  Fuel Efficiencies
\item
  Refine W/S and L/D to be more in line with mission parameters
\end{itemize}
\end{document}% report_template.tex
\documentclass[10pt]{article}
\usepackage[utf8]{inputenc}
\usepackage{graphicx}
\usepackage{amsmath}
\usepackage{siunitx}
\usepackage{booktabs}
\usepackage[margin=0.3in]{geometry}  % Very narrow margins

% Make sections more compact
\usepackage{titlesec}
\titlespacing*{\section}{0pt}{1.2ex plus 1ex minus .2ex}{0.5ex plus .2ex}
\titlespacing*{\subsection}{0pt}{1ex plus 1ex minus .2ex}{0.3ex plus .2ex}

% Adjust list spacing
\usepackage{enumitem}
\setlist{nosep}  % Remove vertical spacing between items
\setlist{leftmargin=*}  % Align with left margin

% Reduce paragraph spacing
\setlength{\parskip}{0.2ex plus 0.1ex minus 0.1ex}
\setlength{\parindent}{0pt}

% Make title more compact
\usepackage{titling}
\setlength{\droptitle}{-1em}  % Move title up
\pretitle{\begin{center}\Large}  % Smaller title
\posttitle{\par\end{center}}
\preauthor{\begin{center}\normalsize}  % Smaller author
\postauthor{\end{center}}
\predate{\begin{center}\normalsize}  % Smaller date
\postdate{\end{center}}

\title{Aerospace Senior Project Weekly Report}
\author{Team Gamma}
\date{Week 1 - January 21, 2025}

\begin{document}
\maketitle

\section{Progress Made}\label{progress-made}

\subsection{Product Breakdown
Structure}\label{product-breakdown-structure}

\begin{itemize}
\tightlist
\item
  We began by forming an initial product Breakdown
\item
  We then used this breakdown to begin researching requirements
\end{itemize}

\subsection{Research Topics Summary}\label{research-topics-summary}

We began researching:

\begin{itemize}
\tightlist
\item
  FAA Requlations on:

  \begin{itemize}
  \tightlist
  \item
    Runways
  \item
    Hangers
  \item
    Weather conditions
  \item
    Stability
  \item
    Safety (fire suppresion and cabin dimensions)
  \end{itemize}
\item
  Comparable Jets

  \begin{itemize}
  \tightlist
  \item
    The A380 was used for many preliminary calculations, as the largest
    currently operated passenger jet.
  \end{itemize}
\item
  Geometry using ch.~5\&6 of Raymer \#\# Requirements
\item
  Since we are designing a large transport vehicle, the 14 CFR 25
  (airworthiness standards for transport category airplanes) was used to
  determine many of the FAA requirements for the plane.
\item
  Mission requirements were also categorized.
\item
  Other ``customer requirements'' were drafted from mission requirements
  and regulations. These will be used to guide trade studies for the
  determination of design requirements.
\item
  Requirements were drafted in the following categories:

  \begin{itemize}
  \tightlist
  \item
    Stability
  \item
    Structure
  \item
    Passenger loading
  \item
    Weight
  \item
    Performance
  \item
    Operational Internal Components
  \item
    Passenger components: Windows, seats, bathrooms, food preparation
  \end{itemize}
\end{itemize}

\subsection{Weight Calculations}\label{weight-calculations}

\begin{itemize}
\tightlist
\item
  Crew weight was found to be \textasciitilde4,818 lbs
\item
  Payload weight was found to be \textasciitilde331,660 lbs
\item
  Total weight was found to be \textasciitilde2,233,541 lbs
\end{itemize}

\subsection{Additional Work}\label{additional-work}

Mission diagram, project breakdown structure diagram and preliminary
function block diagrams were created.

\section{Next Week's Objectives}\label{next-weeks-objectives}

\subsection{Continued Research}\label{continued-research}

Next week, we aim to continue researching the above topics. Specific
additional research includes:

\begin{itemize}
\tightlist
\item
  Water Storage
\item
  Takeoff thrust determination (this will aid in specifying many engine
  requirements)
\item
  Taining requirements
\end{itemize}

\subsection{Trade Studies}\label{trade-studies}

We aim to begin conducting the following trade studies:

\begin{itemize}
\tightlist
\item
  Cabin and Seat configuration
\item
  Cruise thrust requirements
\item
  Takeoff/Landing thrust requirements
\item
  Alternative fuselage designs (double fuselage, flying wing, double
  decker, etc.)
\item
  Wing geometry
\item
  Fuel Efficiencies
\item
  Refine W/S and L/D to be more in line with mission parameters
\end{itemize}

\end{document}

% presentation_template.tex
\documentclass{beamer}
\usetheme{Madrid}
\usecolortheme{whale}
\usepackage{graphicx}
\usepackage{amsmath}
\usepackage{siunitx}

\title{Aerospace Senior Project Progress}
\author{Team Gamma}
\date{Week 1 - January 21, 2025}

\begin{document}

\frame{\titlepage}

\frame{
\frametitle{Contents}
\tableofcontents
}

\section{Progress Made}\label{progress-made}

\subsection{Product Breakdown
Structure}\label{product-breakdown-structure}

\begin{itemize}
\tightlist
\item
  We began by forming an initial product Breakdown
\item
  We then used this breakdown to begin researching requirements
\end{itemize}

\subsection{Research Topics Summary}\label{research-topics-summary}

We began researching:

\begin{itemize}
\tightlist
\item
  FAA Requlations on:

  \begin{itemize}
  \tightlist
  \item
    Runways
  \item
    Hangers
  \item
    Weather conditions
  \item
    Stability
  \item
    Safety (fire suppresion and cabin dimensions)
  \end{itemize}
\item
  Comparable Jets

  \begin{itemize}
  \tightlist
  \item
    The A380 was used for many preliminary calculations, as the largest
    currently operated passenger jet.
  \end{itemize}
\item
  Geometry using ch.~5\&6 of Raymer \#\# Requirements
\item
  Since we are designing a large transport vehicle, the 14 CFR 25
  (airworthiness standards for transport category airplanes) was used to
  determine many of the FAA requirements for the plane.
\item
  Mission requirements were also categorized.
\item
  Other ``customer requirements'' were drafted from mission requirements
  and regulations. These will be used to guide trade studies for the
  determination of design requirements.
\item
  Requirements were drafted in the following categories:

  \begin{itemize}
  \tightlist
  \item
    Stability
  \item
    Structure
  \item
    Passenger loading
  \item
    Weight
  \item
    Performance
  \item
    Operational Internal Components
  \item
    Passenger components: Windows, seats, bathrooms, food preparation
  \end{itemize}
\end{itemize}

\subsection{Weight Calculations}\label{weight-calculations}

\begin{itemize}
\tightlist
\item
  Crew weight was found to be \textasciitilde4,818 lbs
\item
  Payload weight was found to be \textasciitilde331,660 lbs
\item
  Total weight was found to be \textasciitilde2,233,541 lbs
\end{itemize}

\subsection{Additional Work}\label{additional-work}

Mission diagram, project breakdown structure diagram and preliminary
function block diagrams were created.

\section{Next Week's Objectives}\label{next-weeks-objectives}

\subsection{Continued Research}\label{continued-research}

Next week, we aim to continue researching the above topics. Specific
additional research includes:

\begin{itemize}
\tightlist
\item
  Water Storage
\item
  Takeoff thrust determination (this will aid in specifying many engine
  requirements)
\item
  Taining requirements
\end{itemize}

\subsection{Trade Studies}\label{trade-studies}

We aim to begin conducting the following trade studies:

\begin{itemize}
\tightlist
\item
  Cabin and Seat configuration
\item
  Cruise thrust requirements
\item
  Takeoff/Landing thrust requirements
\item
  Alternative fuselage designs (double fuselage, flying wing, double
  decker, etc.)
\item
  Wing geometry
\item
  Fuel Efficiencies
\item
  Refine W/S and L/D to be more in line with mission parameters
\end{itemize}

\end{document}
